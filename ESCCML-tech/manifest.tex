\documentclass[11pt, a4paper]{article}
\usepackage{amsmath, amssymb, hyperref, geometry, booktabs}
\geometry{total={170mm,257mm}, left=25mm, top=25mm}
\title{\textbf{ESCCML Study: M1. Manifest and Ledger Ingestion}}
\author{ESCCML Technical Working Group}
\date{Version 1.0}

\begin{document}
\maketitle

\begin{abstract}
This study introduces the ESCCML (Epoch-Split Codec Convergence Manifest Ledger) standard,
focusing on the Manifest and Ledger ingestion pipeline. Building on Rotocoin’s Phase 0 validation,
we demonstrate how deterministic serialization, lock-free WAL ingestion, and codec-native processing
enable validator-class throughput on commodity hardware with ultra-low energy cost.
\end{abstract}

\section{Introduction}
Traditional blockchains rely on block-confirmation consensus and VM-based execution, which
introduce latency and energy overhead. ESCCML redefines the ledger substrate by introducing
\emph{Manifest-sealed finality} and a \emph{lock-free ingestion pipeline}, enabling deterministic,
auditable, and energy-efficient state transitions.

\subsection{Design Goals}
\begin{itemize}
    \item Deterministic replayability across hardware architectures.
    \item Sub-second finality via Manifest sealing.
    \item Ultra-low energy per transaction ($\sim 10^{-12}$ kWh/tx).
    \item Auditability through epoch-indexed anchoring.
\end{itemize}

\section{The Manifest: Atomic Unit of State Transition}
\subsection{Structure and Sealing}
A Manifest is a cryptographically sealed, immutable structure generated by an Executor worker.
Its digest $\mathbf{M_D}$ must contain:
\begin{enumerate}
    \item Merkle Root ($\mathbf{M_{root}}$)
    \item Validator Signature ($\mathbf{\Sigma_V}$)
    \item Codec Version Hash ($\mathbf{H_C}$)
    \item Timestamp ($\mathbf{T_E}$)
\end{enumerate}

\subsection{Finality Model}
Unlike blockchains where finality is probabilistic or delayed, ESCCML finality is
\textbf{Manifest-sealed}. Once a Manifest is sealed and ingested, the state transition is final.

\section{The Ledger: Lock-Free Ingestion Pipeline}
\subsection{Architecture}
\begin{itemize}
    \item Per-thread buffers for local writes.
    \item Asynchronous flusher for batched WAL commits.
    \item Lock-free design to eliminate mutex contention.
\end{itemize}

\subsection{Performance Benchmarks}
Phase 0 validation demonstrated:
\begin{itemize}
    \item $>$4M TPS on commodity hardware.
    \item Manifest flush $p95 < 2.5$ ms under durable workloads.
    \item Energy ceiling $\sim 1.4 \times 10^{-12}$ kWh/tx.
\end{itemize}

\section{Codec Integration}
The channelized codec (Y/Cb/Cr-style) ensures:
\begin{itemize}
    \item High-salience values (balances, transfers) are encoded with exact precision.
    \item Metadata and proofs are compressed in lower-salience channels.
    \item Zero-copy Merkle hashing and partial decoding for light clients.
\end{itemize}

\section{Epoch-Split Anchoring}
Epoch-indexed anchoring provides:
\begin{itemize}
    \item Immutable, time-ordered state convergence.
    \item Forensic auditability across latency zones.
    \item Support for asynchronous validation (e.g., Earth–Mars coordination).
\end{itemize}

\section{Economic Implications}
\begin{itemize}
    \item Fee routing and Piggy Bank accumulators are sealed in manifests.
    \item Stake-to-throughput ratio enforces validator realism.
    \item Privilege decay prevents stake concentration.
\end{itemize}

\section{Discussion and Future Work}
\begin{itemize}
    \item Integration with Axum async servers for declogging ingestion.
    \item Friend-batch convergence and gossip pre-share protocols.
    \item Governance scaffolding (seat leasing, tax routing, coupon renewal).
\end{itemize}

\section{Conclusion}
The ESCCML ingestion model demonstrates that a ledger does not require high energy draw
or heavyweight consensus to achieve validator-class throughput. By combining codec-native
serialization, lock-free WAL ingestion, and manifest-sealed finality, ESCCML establishes a
new substrate for infra-native value streams.

\end{document}
